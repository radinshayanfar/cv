%-------------------------------------------------------------------------------
%	SECTION TITLE
%-------------------------------------------------------------------------------
\cvsection{Projects}


%-------------------------------------------------------------------------------
%	CONTENT
%-------------------------------------------------------------------------------
\begin{cventries}

%---------------------------------------------------------  
  \cventry
    {Python, Keras, PyTorch, Genrative Models} % Skills
    {Generating Handwritten Persian Characters with Generative Models} % Project
    {} % Empty location
    {2022-PRESENT} % Date
    {
      \begin{cvitems} % Description(s) bullet points
      	\item {Analysis of the recently introduced \textbf{generative models} ability to represent and reconstruct individual Persian handwritten characters shape. The models under study are \href{https://arxiv.org/abs/1406.2661}{\textbf{GAN}} and its variations (like \href{https://arxiv.org/abs/1511.06434}{\textbf{DCGAN}} and \href{https://arxiv.org/abs/1704.00028}{\textbf{WGAN-GP}}), \href{https://arxiv.org/abs/1312.6114}{\textbf{VAE}}, and \href{https://arxiv.org/abs/1605.08803}{\textbf{Real NVP} (Normalizing Flows)}.}
      	\item {This is my \textbf{BSc. project} under supervision of Prof. A. Nickabadi.}
      \end{cvitems}
    }

%---------------------------------------------------------  
  \cventry
    {Python} % Skills
    {\href{https://github.com/radinshayanfar/AUT-IR}{Infromation Retrieval Engine}} % Project
    {} % Empty location
    {2022} % Date
    {
      \begin{cvitems} % Description(s) bullet points
      	\item {Implementation of an information retrieval system using Python programming language as part of the university's Information Retrieval course. A \textbf{positional inverted index} is utilized at the core of the system.}
      	\item {The engine can respond to queries in boolean and ranked manners. The ranked approach uses a \textbf{tf-idf} scoring technique to rank the results.}
      \end{cvitems}
    }

%---------------------------------------------------------  
  \cventry
    {Python, NumPy, Signals and Systems} % Skills
    {Music Identification} % Project
    {} % Empty location
    {Spring 2021} % Date
    {
      \begin{cvitems} % Description(s) bullet points
      	\item {\textbf{Design} and development of a music identification system using \textbf{Fast Fourier Transform} (FFT). The system identifies songs using an audio fingerprint based on songs' \textbf{spectrogram}, which is calculated from scratch by calculating individual FFTs. The fingerprint is obtained by filtering out very low and very high frequencies, binning the frequency-domain of the spectrogram, and saving frequencies with maximum magnitude in each bin as the representative frequency of each time slice. The fingerprints of signals are compared using a naïve similarity metric.}
        \item {Designed and assigned to students as the Signals and Systems course final project by me as the teaching assistant.}
      \end{cvitems}
    }

%---------------------------------------------------------  
  \cventry
    {Python, Keras, OpenCV} % Skills
    {Captcha Solver} % Project
    {} % Empty location
    {2019 - 2020} % Date
    {
      \begin{cvitems} % Description(s) bullet points
      	\item {Practiced solving CAPTCHA images using \textbf{image processing} techniques (using \textbf{OpenCV}) and \textbf{neural networks} (using \textbf{Keras}).}
      	\item {Achieved 96\% accuracy on simple 5-letters CAPTCHAs.}
      \end{cvitems}
    }
    
%---------------------------------------------------------  
  \cventry
    {Laravel, MySQL, Redis, REST API} % Skills
    {Habco} % Project
    {} % Empty location
    {2021} % Date
    {
      \begin{cvitems} % Description(s) bullet points
      	\item {Development of a \textbf{REST API} for Habco (a Canadian startup) application, a medical application for patients, doctors, pharmacists, and nurses.}
      	\item {In habco, patients can choose doctors and nurses. Doctors can write prescriptions for the patient and the nurses can write instructions for them. Patients can also send prescriptions to pharmacies and track its status. There's also a drug stock management panel for pharmacists.}
      	\item {The API is written in PHP, using \textbf{Laravel} framework. It uses \textbf{SMS code verification} for authentication.}
      \end{cvitems}
    }

%---------------------------------------------------------  
%  \cventry
%    {PHP, HTML, CSS, Bootstrap, Sass, JavaScript} % Skills
%    {\href{https://samcode.allamehelli3.ir/}{SamCode Website}} % Project
%    {} % Empty location
%    {2020} % Date
%    {
%      \begin{cvitems} % Description(s) bullet points
%      	\item {Design and development of SamCode programming contest website (Held by \href{http://helli3school.ir}{Allameh Helli 3 Junior High School}).}
%      	\item {The website uses \textbf{Datalife Engine (DLE) CMS} and my work involved design and development of a \textbf{new template} for the CMS. The template is created by merging a simple landing page with DLE's default template.}
%      \end{cvitems}
%    }

%---------------------------------------------------------  
%  \cventry
%    {C++, OpenCV} % Skills
%    {Ping-Pong Ball Tracker} % Project
%    {} % Empty location
%    {2014} % Date
%    {
%      \begin{cvitems} % Description(s) bullet points
%        \item {Development of a ping-pong game ball tracker using C++ and OpenCV features.}
%      \end{cvitems}
%    }

%---------------------------------------------------------
\end{cventries}
