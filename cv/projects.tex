%-------------------------------------------------------------------------------
%	SECTION TITLE
%-------------------------------------------------------------------------------
\cvsection{Projects}


%-------------------------------------------------------------------------------
%	CONTENT
%-------------------------------------------------------------------------------
\begin{cventries}

%---------------------------------------------------------  
\cventry
  {PyTorch, Transformer, Machine Translation, BPE Tokenization} % Skills
  {\href{https://github.com/radinshayanfar/transformer}{Transformer from Scratch!}} % Project
  {} % Location
  {Summer 2025} % Date
  {
    \begin{cvitems}
      \item {Implemented the original \textbf{Transformer architecture} in PyTorch, supporting \textbf{encoder-decoder}, \textbf{encoder-only}, and \textbf{decoder-only} variants.}
      \item {Built a complete translation system for the \textbf{WMT dataset}, including data loading, BPE tokenization, batching, masking, training loop, and evaluation.}
      \item {Developed the system solely from the original paper, without referencing any external implementations.}
      \item {Structured the implementation for extensibility to other tasks such as text generation and visual transformer (ViT) using the same architectural base.}
    \end{cvitems}
  }

%---------------------------------------------------------  
\cventry
    {PyTorch, SpeechBrain, Voice Activity Detection, Speaker Verification} % Skills
    {\href{https://github.com/radinshayanfar/speechbrain}{Speaker-specific Voice Activity Detection with ECAPA-TDNN Embeddings}} % Project
    {} % Location
    {Fall 2024} % Date
    {
      \begin{cvitems}
        \item {Proposed and implemented a novel \textbf{Speaker-Specific Voice Activity Detection (SVAD)} framework integrating \textbf{ECAPA-TDNN embeddings} into a CRDNN-based VAD.}
        \item {Extended the SpeechBrain toolkit to support conditioning on \textbf{target speaker embeddings} by modifying model internals and data pipelines.}
        \item {Engineered robust training and augmentation pipelines, including \textbf{embedding-level noise injection} to improve generalization under speaker-limited data.}
        \item {Evaluated the system on the \textbf{AMI dataset}, achieving strong performance and low Diarization Error Rate under real-time and multi-speaker conditions.}
      \end{cvitems}
    }
    
%---------------------------------------------------------  
\cventry
    {PyTorch, Transformers, Sentiment Analysis} % Skills
    {\href{https://github.com/aghassel/NLP-Sentiment-Analysis/}{Comparative Analysis of Transformer-based Models for Sentiment Analysis}} % Project
    {} % Location
    {Fall 2023} % Date
    {
      \begin{cvitems}
        \item {Fine-tuned a \textbf{pre-trained GPT-2 model} for multi-label emotion and binary sentiment classification on the \textbf{GoEmotions} and \textbf{Sent140} datasets using \textbf{PyTorch}.}
        \item {Adapted GPT-2 with a classification head and trained the model with regularization techniques (e.g., \textbf{dropout}) to mitigate overfitting and improve generalization, demonstrating GPT-2's competitive capability compared to BERT and ELMo.}
        \item {Collaborated on a comprehensive analysis of Transformer-based architectures, providing insights on the strengths and limitations of decoder-only models like GPT-2 in classification tasks.}
      \end{cvitems}
    }

%---------------------------------------------------------  
\cventry
    {PyTorch, Reinforcement Learning, DRQN} % Skills
    {Deep Recurrent Q-Network Agent for a 3D FPS Game} % Project
    {} % Location
    {Fall 2023} % Date
    {
      \begin{cvitems}
        \item {Reimplemented a \textbf{\href{https://arxiv.org/pdf/1609.05521}{Deep Recurrent Q-Network (DRQN)}} from scratch to train an autonomous agent in a 3D FPS environment (VizDoom), based on visual observations.}
        \item {Integrated \textbf{LSTM layers} to handle partial observability and improve temporal context in decision making.}
        \item {Enhanced the architecture by incorporating \textbf{game feature prediction} as an auxiliary task to boost representation learning and agent performance.}
      \end{cvitems}
    }


%---------------------------------------------------------  
  \cventry
    {Python, Keras, PyTorch, Genrative Models} % Skills
    {Generating Handwritten Persian Characters with Generative Models} % Project
    {} % Empty location
    {Feb. 2022 - Apr. 2023} % Date
    {
      \begin{cvitems} % Description(s) bullet points
      	\item {Evaluated and analyzed  \textbf{generative models} ability to represent and reconstruct individual Persian handwritten characters. The studied models were \href{https://arxiv.org/abs/1406.2661}{\textbf{GAN}} and its variations (like \href{https://arxiv.org/abs/1511.06434}{\textbf{DCGAN}} and \href{https://arxiv.org/abs/1704.00028}{\textbf{WGAN-GP}}), \href{https://arxiv.org/abs/1312.6114}{\textbf{VAE}}, \href{https://arxiv.org/abs/1605.08803}{\textbf{Real NVP} (Normalizing Flows)}, and \href{https://arxiv.org/abs/2006.11239}{\textbf{DDPM}}.}
      	\item{Trained a Real NVP model and employed a \textit{novel technique} to \textbf{replicate personal handwriting styles} based on sample data. Final \textbf{BSc.~project} under Prof.~A.~Nickabadi.}
      \end{cvitems}
    }

  \cventry
    {Python} % Skills
    {\href{https://github.com/radinshayanfar/AUT-IR}{Information Retrieval Engine}} % Project
    {} % Empty location
    {2022} % Date
    {
      \begin{cvitems} % Description(s) bullet points
      	\item {Built a text-based search engine with support for \textbf{Boolean} and \textbf{TF-IDF ranked retrieval}, using a positional inverted index in \textbf{Python}.}
      \end{cvitems}
    }
    
  \cventry
    {Laravel, MySQL, Redis, REST API} % Skills
    {Habco} % Project
    {} % Empty location
    {2021} % Date
    {
      \begin{cvitems} % Description(s) bullet points
      	\item {Developed a \textbf{RESTful API} using \textbf{Laravel} for Habco, a Canadian healthcare startup serving patients, doctors, nurses, and pharmacists, enabling features like prescription sharing, drug inventory, and SMS-based authentication.}
      \end{cvitems}
    }

%---------------------------------------------------------  
  \cventry
    {Python, NumPy, Signals and Systems} % Skills
    {Music Identification} % Project
    {} % Empty location
    {Spring 2021} % Date
    {
      \begin{cvitems} % Description(s) bullet points
      	\item {\textbf{Designed and implemented} a music identification and discovery system using \textbf{Fast Fourier Transform (FFT)} to generate audio fingerprints from scratch. The system computes a song's \textbf{spectrogram} by windowing the time-domain signal and calculating successive FFTs.}
        \item {Constructed fingerprints by: (1) filtering out low and high-frequency noise, (2) binning the spectrogram's frequency axis, and (3) extracting peak-magnitude frequencies from each bin as time-slice representatives. Fingerprints are compared using a custom similarity metric.}
        \item {Developed and assigned this project as the final assignment for the undergraduate \textbf{Signals and Systems} course in my role as teaching assistant.}
      \end{cvitems}
    }

%---------------------------------------------------------  
  \cventry
    {PHP, HTML, CSS, Bootstrap, Sass, JavaScript} % Skills
    {\href{https://samcode.allamehelli3.ir/}{SamCode Website}} % Project
    {} % Empty location
    {2020} % Date
    {
      \begin{cvitems} % Description(s) bullet points
      \item {Designed and developed the official website for \textbf{SamCode}, a programming contest hosted by \href{http://helli3school.ir}{Allameh Helli 3 Junior High School}.}
      \item {Customized the \textbf{Datalife Engine (DLE) CMS} by building a new theme that integrated a modern landing page with the system’s default template.}
      \end{cvitems}
    }

%---------------------------------------------------------  
  \cventry
    {Python, Keras, OpenCV} % Skills
    {Captcha Solver} % Project
    {} % Empty location
    {2019 - 2020} % Date
    {
      \begin{cvitems} % Description(s) bullet points
        \item {Built a system to solve CAPTCHA images using \textbf{OpenCV}-based image preprocessing and a \textbf{neural network} model built with \textbf{Keras}.}
        \item {Collected and annotated training data using a custom-built web-based annotation UI.}
        \item {Trained a deep learning model on the labeled dataset and achieved \textbf{96\% accuracy} on 5-character alphanumeric CAPTCHAs.}
      \end{cvitems}
    }

%---------------------------------------------------------
  \cventry
    {PHP, MySQL, Telegram Bot API} % Skills
    {\href{https://github.com/radinshayanfar/RJBot}{RJBot}} % Project
    {} % Empty location
    {2019 - 2020} % Date
    {
      \begin{cvitems} % Description(s) bullet points
          \item {Developed an open-source \textbf{Telegram bot} for searching and downloading media from RadioJavan.com, using PHP and the Telegram Bot API.}
          \item {Supported various media types including music, videos, albums, and podcasts.}
          \item {Designed to practice and demonstrate \textbf{object-oriented programming} principles.}
      \end{cvitems}
    }

% %---------------------------------------------------------    
%   \cventry
%     {Laravel, MySQL} % Skills
%     {Chaladz Design} % Project
%     {} % Empty location
%     {2019} % Date
%     {
%       \begin{cvitems} % Description(s) bullet points
%         \item {Back-end development of Chaladz Design's online shop, written in Laravel framework.} % (Discontinued after a month).
% 		\item {The website has admin panel, cart, order tracking, and other regular features of an online shop.}
%       \end{cvitems}
%     }

%---------------------------------------------------------    
  \cventry
    {Java, Swing} % Skills
    {\href{https://github.com/radinshayanfar/Jpotify}{Jpotify}} % Project
    {} % Empty location
    {2019} % Date
    {
      \begin{cvitems} % Description(s) bullet points
          \item {Created a desktop music player in Java using \textbf{Swing} for the graphical interface.}
          \item {Implemented features such as playlist management and music sharing over a local network.}
      \end{cvitems}
    }

%%---------------------------------------------------------
%  \cventry
%    {HTML, JavaScript, CSS} % Skills
%    {\href{https://konkur98.ga}{Konkur98}} % Project
%    {} % Empty location
%    {2017} % Date
%    {
%      \begin{cvitems} % Description(s) bullet points
%      	\item {Front-end development of a countdown website for \href{https://en.wikipedia.org/wiki/Iranian_University_Entrance_Exam}{Nationwide University Entrance Exam of Iran}.}
%      \end{cvitems}
%    }

%---------------------------------------------------------  
  \cventry
    {C++, OpenCV} % Skills
    {Ping-Pong Ball Tracker} % Project
    {} % Empty location
    {2014} % Date
    {
      \begin{cvitems} % Description(s) bullet points
        \item {Developed a real-time ping-pong ball tracking system using \textbf{OpenCV} and \textbf{C++}.}
      \end{cvitems}
    }

%---------------------------------------------------------  
  \cventry
    {Delphi} % Skills
    {Ticket to Ride} % Project
    {} % Empty location
    {2013} % Date
    {
      \begin{cvitems} % Description(s) bullet points
       \item {Built a two-player graphical board game in Delphi, inspired by \textit{Ticket to Ride}.}
      \item {Completed as part of a “\emph{Pajooheshi}” (research) class project in 8th grade.}
      \end{cvitems}
    }

%---------------------------------------------------------
\end{cventries}
