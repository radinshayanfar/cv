%-------------------------------------------------------------------------------
%	SECTION TITLE
%-------------------------------------------------------------------------------
\cvsection{Projects}


%-------------------------------------------------------------------------------
%	CONTENT
%-------------------------------------------------------------------------------
\begin{cventries}

%---------------------------------------------------------  
  \cventry
    {Python, Keras, PyTorch, Genrative Models} % Skills
    {Generating Handwritten Persian Characters with Generative Models} % Project
    {} % Empty location
    {2022-PRESENT} % Date
    {
      \begin{cvitems} % Description(s) bullet points
      	\item {Analysis of the recent \textbf{generative models} ability to represent and reconstruct individual Persian handwritten characters shape. The models under study are \href{https://arxiv.org/abs/1511.06434}{\textbf{DCGAN}} and \href{https://arxiv.org/abs/1312.6114}{\textbf{VAE}}.}
      	\item {This is my BSc. project under supervision of Prof. A. Nickabadi.}
      \end{cvitems}
    }

  \cventry
    {Python} % Skills
    {\href{https://github.com/radinshayanfar/AUT-IR}{Infromation Retrieval Engine}} % Project
    {} % Empty location
    {2022} % Date
    {
      \begin{cvitems} % Description(s) bullet points
      	\item {Implementation of an information retrieval system using Python programming language as part of the university's Information Retrieval course. A \textbf{positional inverted index} is utilized at the core of the system.}
      	\item {The engine can respond to queries in two manners; boolean and ranked. The ranked approach uses a \textbf{tf-idf} scoring technique to rank the results.}
      \end{cvitems}
    }
    
  \cventry
    {Laravel, MySQL, Redis, REST API} % Skills
    {Habco} % Project
    {} % Empty location
    {2021} % Date
    {
      \begin{cvitems} % Description(s) bullet points
      	\item {Development of a \textbf{REST API} for Habco (a Canadian startup), a medical application for patients, doctors, pharmacists, and nurses.}
      	\item {In habco, patients can choose doctors and nurses. The chosen doctors can write prescriptions for the patient and the nurses can write instructions for them. Patients can also send prescriptions to pharmacies and track its status. There's also a drug stock management panel for pharmacists.}
      	\item {The API is written is PHP, using \textbf{Laravel} framework. It uses \textbf{SMS code verification} for authentication.}
      \end{cvitems}
    }

%---------------------------------------------------------  
  \cventry
    {PHP, HTML, CSS, Bootstrap, Sass, JavaScript} % Skills
    {\href{https://samcode.allamehelli3.ir/}{SamCode Website}} % Project
    {} % Empty location
    {2020} % Date
    {
      \begin{cvitems} % Description(s) bullet points
      	\item {Design and development of SamCode programming contest website (Hold by \href{http://helli3school.ir}{Allameh Helli 3 Junior High School}).}
      	\item {The website uses \textbf{Datalife Engine (DLE) CMS} and my work involves design and development of a \textbf{new template} for the CMS. The template is created by merging a simple landing page with DLE's default template.}
      \end{cvitems}
    }

%---------------------------------------------------------  
  \cventry
    {Python, Keras, OpenCV} % Skills
    {Captcha Solver} % Project
    {} % Empty location
    {2019 - 2020} % Date
    {
      \begin{cvitems} % Description(s) bullet points
      	\item {Some research on solving CAPTCHA images using \textbf{image processing} techniques (using \textbf{OpenCV}) and \textbf{neural networks} (using \textbf{Keras}).}
      	\item {Achieved 96\% accuracy on simple 5-letters CAPTCHAs.}
      \end{cvitems}
    }

%---------------------------------------------------------
  \cventry
    {PHP, MySQL, Telegram Bot API} % Skills
    {\href{https://github.com/radinshayanfar/RJBot}{RJBot}} % Project
    {} % Empty location
    {2019 - 2020} % Date
    {
      \begin{cvitems} % Description(s) bullet points
        \item {Development of a Telegram bot for searching and downloading media from the RadioJavan.com website, written in PHP.}
		    \item {The bot supports almost all types of media such as music, video, album, and podcast.}
		    \item {Developed for the purpose of training \textbf{OOP concepts}.}
      \end{cvitems}
    }

%---------------------------------------------------------    
  \cventry
    {Laravel, MySQL} % Skills
    {Chaladz Design} % Project
    {} % Empty location
    {2019} % Date
    {
      \begin{cvitems} % Description(s) bullet points
        \item {Back-end development of Chaladz Design's online shop, written in Laravel framework.} % (Discontinued after a month).
		\item {The website has admin panel, cart, order tracking, and other regular features of an online shop.}
      \end{cvitems}
    }

%---------------------------------------------------------    
  \cventry
    {Java, Swing} % Skills
    {\href{https://github.com/radinshayanfar/Jpotify}{Jpotify}} % Project
    {} % Empty location
    {2019} % Date
    {
      \begin{cvitems} % Description(s) bullet points
        \item {Development of a graphical music player, written in Java with use of Swing library for UI design.}
		\item {The player has features like playlists and music sharing with friends over network.}
      \end{cvitems}
    }

%---------------------------------------------------------
  \cventry
    {HTML, JavaScript, CSS} % Skills
    {\href{https://konkur98.ga}{Konkur98}} % Project
    {} % Empty location
    {2017} % Date
    {
      \begin{cvitems} % Description(s) bullet points
      	\item {Front-end development of a countdown website for Iranian university entrance exam.}
      \end{cvitems}
    }

%---------------------------------------------------------  
  \cventry
    {C++, OpenCV} % Skills
    {Ping-Pong Ball Tracker} % Project
    {} % Empty location
    {2014} % Date
    {
      \begin{cvitems} % Description(s) bullet points
        \item {Development of a ping-pong game ball tracker using C++ and OpenCV features.}
      \end{cvitems}
    }

%---------------------------------------------------------  
  \cventry
    {Delphi} % Skills
    {Ticket to Ride} % Project
    {} % Empty location
    {2013} % Date
    {
      \begin{cvitems} % Description(s) bullet points
        \item {Graphical implementation of a 2 players game using Delphi.}
      \end{cvitems}
    }

%---------------------------------------------------------
\end{cventries}
